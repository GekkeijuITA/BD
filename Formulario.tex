\documentclass[10pt]{article}
\usepackage[italian]{babel}
\usepackage{geometry}
\usepackage{amsmath}
\usepackage{amssymb}

\geometry{margin=2cm}
\let\olditemize\itemize
\renewcommand\itemize{\olditemize\setlength\itemsep{0em}}

\title{Formulario}
\date{}

\begin{document}
\maketitle
\begin{table}[h]
    \centering
    \begin{tabular}{|c|c|c|c|}
        \hline
        \textbf{Op.} & \textbf{Cond.} & \textbf{Lett.} & \textbf{SQL}\\
        \hline
        $\Pi_{A}(R)$ &  & isolo colonne, senza duplicati & \texttt{SELECT DISTINCT A FROM R}\\
        $\sigma_{F}(R)$ &  & isolo righe che soddisfano $F$ & \texttt{SELECT * FROM R WHERE F}\\
        $R \times S$ & schemi disgiunti & tutte le combinazioni di $R$ e $S$ & \texttt{R CROSS JOIN S}\\
        $R \cup S$ & stesso schema & tuple in $R$ \textbf{o} $S$ & \texttt{R UNION S}\\
        $R \cap S$ & stesso schema & tuple in $R$ \textbf{e} in $S$ & \texttt{R INTERSECT S}\\
        $R - S$ & stesso schema e grado & tuple in $R$ ma \textbf{non in} $S$ & \texttt{EXCEPT} o \texttt{NOT IN/NOT EXISTS}\\
        $R \bowtie_{F} S$ & schemi disgiunti & prodotto cartesiano con selezione & \texttt{R JOIN S ON F}\\
        $R \bowtie S$ & almeno un attributo in comune & prodotto cartesiano con selezione & \texttt{R NATURAL JOIN S}\\
        $R \div S$ & almeno un attributo in comune & tuple in $R$ che compaiono in $S$ & \\
        \hline
    \end{tabular}
\end{table}
\textbf{Tips:}
\begin{itemize}
    \item La divisione $R \div S$ si può anche scrivere come $\Pi_{D}(R) - \Pi_{D}((\Pi_{D}(R)\times S)-R)$
    \item Parti sempre dalle sotto-query
    \item La divisione usala quando nella richiesta c'è una condizione del tipo "tutti"
    \item Se usi un operatore di aggregazione (\texttt{COUNT}, \texttt{SUM}, \texttt{AVG}, \texttt{MAX}, \texttt{MIN}), devi usare \texttt{GROUP BY} con lo stesso attributo usato nell'operatore e puoi usare \texttt{HAVING}
\end{itemize}
\textbf{Normalizzazione tips:}
\begin{itemize}
    \item $X \rightarrow Y$: per una stessa $X$ non ci sono $Y$ diverse
    \item Se un attributo non compare mai a destra allora fa parte sicuramente della chiave
    \item E' BCNF se per ogni $A\rightarrow B$, $A$ chiave e $B \nsubseteq A$
    \item E' 3NF se $A\rightarrow B$ con $A$ chiave/superchiave \textbf{oppure} $B$ attributo primo
\end{itemize}
\textbf{Decomposizione in 3NF:}
\begin{itemize}
    \item Individuare le chiavi candidate \begin{itemize}
        \item Scrivere tutte le $X^{+}$ per ogni attributo a sinistra della freccia delle dipendenze funzionali fornite
        \item Se ci sono attributi multipli, in $X^{+}$ vanno le dipendenze dei singoli attributi e le dipendenze degli attributi multipli
        \item Ricorda di mettere anche la dipendenze derivate seguendo la regola sopra
        \item Le chiavi candidate sono tutte quelle $X^{+}$ che contengono tutti gli attributi della tabella
    \end{itemize}
    \item Scompongo in tabelle delle dipendenze usando le dipendenze funzionali
    \item Guardo se almeno una delle tabella ha come chiave la chiave della tabella originale
    \item Se la chiave si trova a sinistra è anche BCNF
\end{itemize}
\end{document}

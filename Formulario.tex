\documentclass[10pt]{article}
\usepackage[italian]{babel}
\usepackage{geometry}
\usepackage{amsmath}
\usepackage{amssymb}

\geometry{margin=2cm}
\let\olditemize\itemize
\renewcommand\itemize{\olditemize\setlength\itemsep{0em}}

\author{Lorenzo Vaccarecci}

\title{Formulario}
\date{}

\begin{document}
\maketitle
\section{Algebra Relazionale}
\begin{table}[h]
    \centering
    \begin{tabular}{|c|c|c|}
    \hline
    \textbf{Operatore} & \textbf{Descrizione} & \textbf{Uso}\\
    \hline
    $\sigma_{\theta}$ & Selezione & isolo righe\\
    \hline
    $\pi_{A_{1}, A_{2}, \ldots, A_{n}}$ & Proiezione & isolo colonne\\
    \hline
    $\times$ & Prodotto cartesiano & solo se schemi disgiunti\\
    \hline
    $\cup$ & Unione & solo se stesso schema\\
    \hline
    $\cap$ & Intersezione & solo se stesso schema\\
    \hline
    $-$ & Differenza & solo se stesso schema e grado\\
    \hline
    $\bowtie$ & Join & solo se schemi disgiunti\\
    \hline
    $\div$ & Divisione & query con 'tutto'\\
    \hline
    \end{tabular}
\end{table}
Divisione: Quali sono gli elementi di A che sono correlati a tutti gli elementi di B?
\section{Outer Join}
% Adjusting table font size and row height
\begin{table}[h]
    \centering
    \small % Font size adjustment
    \setlength{\arrayrulewidth}{0.5mm} % Thickness of the table border
    \renewcommand{\arraystretch}{1.2} % Row height
    \begin{tabular}{|c|c|c|c|}
    \hline
    \textbf{Tabella A} & \textbf{Tabella B} & \textbf{Left Outer Join} & \textbf{Right Outer Join} \\
    \hline
    \begin{tabular}{|c|c|}
    \hline
    \textbf{ID} & \textbf{Nome}  \\
    \hline
    1  & Anna  \\
    \hline
    2  & Bruno \\
    \hline
    3  & Carlo \\
    \hline
    \end{tabular} &
    \begin{tabular}{|c|c|}
    \hline
    \textbf{ID} & \textbf{Cognome}  \\
    \hline
    3  & Rossi    \\
    \hline
    4  & Verdi    \\
    \hline
    5  & Bianchi  \\
    \hline
    \end{tabular} &
    \begin{tabular}{|c|c|c|}
    \hline
    \textbf{ID} & \textbf{Nome} & \textbf{Cognome} \\
    \hline
    1  & Anna  & NULL    \\
    \hline
    2  & Bruno & NULL    \\
    \hline
    3  & Carlo & Rossi   \\
    \hline
    \end{tabular} &
    \begin{tabular}{|c|c|c|}
    \hline
    \textbf{ID} & \textbf{Nome} & \textbf{Cognome} \\
    \hline
    3  & Carlo & Rossi    \\
    \hline
    4  & NULL  & Verdi    \\
    \hline
    5  & NULL  & Bianchi  \\
    \hline
    \end{tabular} \\
    \hline
    \end{tabular}
\end{table}

% Adjusting table font size and row height
\begin{table}[h]
    \centering
    \small % Font size adjustment
    \setlength{\arrayrulewidth}{0.5mm} % Thickness of the table border
    \renewcommand{\arraystretch}{1.2} % Row height
    \begin{tabular}{|c|}
    \hline
    \textbf{Full Outer Join} \\
    \hline
    \begin{tabular}{|c|c|c|}
    \hline
    \textbf{ID} & \textbf{Nome} & \textbf{Cognome} \\
    \hline
    1  & Anna  & NULL     \\
    \hline
    2  & Bruno & NULL     \\
    \hline
    3  & Carlo & Rossi    \\
    \hline
    4  & NULL  & Verdi    \\
    \hline
    5  & NULL  & Bianchi  \\
    \hline
    \end{tabular} \\
    \hline
    \end{tabular}
\end{table}    

\section{Normalizzazione}
\end{document}

\documentclass[12pt]{article}
\usepackage[italian]{babel}
\usepackage{geometry}
\usepackage{amsmath}
\usepackage{amssymb}
\usepackage{graphicx}
\usepackage{ulem}

\geometry{margin=2cm}
\newtheorem{definition}{Definizione}
\newtheorem{example}{Esempio}

\title{Continuazione - Chiavi Esterne}
\author{Lorenzo Vaccarecci}
\date{5 Marzo 2024}

\graphicspath{{../Immagini/}}

\begin{document}
\maketitle
\begin{itemize}
    \item \textbf{Chiave esterna}: chiave della relazione riferita
    \item \textbf{Relazione referente}: relazione dove la chiave esterna viene utilizzata \\(\(Y\subseteq U_{R}\) è chiave esterna in $R$ su $R'$)
    \item \textbf{Relazione riferita}: relazione a cui si fa riferimento con la chiave esterna \\(\(Y\subseteq U_{R}'\) chiave per $R'$)
\end{itemize}
\(\forall \text{ istanza di } R,R' \text{ } \forall t \text{ nell'istanza di } R \text{ }\exists t' \text{ nell'istanza di } R' \text{ t.c. } t\left[Y\right]=t'\left[Y'\right]\)
\\ $f$ funzione iniettiva, $Y$ chiave per $R'$, $X$ chiave esterna per $R$: \(X=f(Y)\subseteq U_{R}\)
\\\textbf{Le chiavi esterne le indichiamo con un apice}: Video(\uline{colloc},\(\text{titolo}^\text{FILM}\),\(\text{regista}^\text{FILM}\),tipo)
\end{document}
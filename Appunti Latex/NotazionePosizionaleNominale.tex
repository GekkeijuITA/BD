\documentclass[12pt]{article}
\usepackage[italian]{babel}
\usepackage{geometry}
\usepackage{amsmath}

\geometry{margin=2cm}
\newtheorem{definition}{Definizione}

\title{Notazione Posizionale e Nominale}
\author{Lorenzo Vaccarecci}
\date{27 Febbraio 2024}

\begin{document}
\maketitle
\begin{definition}
    \textbf{Dominio}: un insieme (anche infinito) di valori
\end{definition}
\begin{definition}
    Un insieme è \textbf{non ordinato} e \textbf{non ha duplicati}
\end{definition}
\begin{definition}
    \textbf{Relazione}: sottoinsieme \underline{finito} del prodotto cartesiano di n domini
\end{definition}
\section{Prodotto Cartesiano (Notazione Posizionale)}
\(D_{1},\dots,D_{N}\in \mathcal{D} \) insiemi (anche infiniti) di valori (domini)\\
Prodotto Cartesiano \(D_{1}\times\dots\times D_{N}\)\\
\(\left\{\left(v_{1},\dots ,v_{N}\right)|v_{1}\in D_{1},\dots,v_{N}\in D_{N}\right\} \equiv \left\{t:[l,n]\rightarrow D_{1}\cup\dots\cup D_{n}|t(i)\in D_{i} \text{ } i=l\dots n\right\}\)
\\Tutte le possibili combinazioni
\subsection*{Esempio}
\(D_{1}=\{0,1,2\}\)\\
\(D_{2}=\{d,v\}\)\\
\(D_{1}\times D_{2} = \{(0,d),(0,v),(1,d),(1,v),(2,d),(2,v)\} \neq D_{2}\times D_{1}\)\\
Ha cardinalità 6 e grado 2.
(grado = colonne; cardinalità = righe)
\section{Prodotto Cartesiano (Notazione Nominale)}
\(A_{1},\dots,A_{N}\) nomi di attributi\\
\(\left\{\left(A_{1}:v_{1},\dots ,A_{n}:v_{N}\right)|v_{1}\in D_{1},\dots,v_{N}\in D_{N}\right\} \equiv \left\{t:[A_{1},\dots,A_{n}]\rightarrow D_{1}\cup\dots\cup D_{n}|t(A_{i})\in D_{i} \text{ } i=l\dots n\right\}\)
\section{Valori Nulli}
? NULL \(v_{i}\in D_{i}\cup\{?\}\)
\section{Esempio di esercizio}
\end{document}
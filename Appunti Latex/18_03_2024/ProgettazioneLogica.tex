\documentclass[12pt]{article}
\usepackage[utf8]{inputenc}
\newcommand\preamble{
    \usepackage[italian]{babel}
    \usepackage{geometry}
    \usepackage{amsmath}
    \usepackage{amssymb}
    \usepackage{graphicx}
    \usepackage{ulem}
    \usepackage[dvipsnames]{xcolor}

    \geometry{margin=2cm}
    \let\olditemize\itemize
    \renewcommand\itemize{\olditemize\setlength\itemsep{0em}}
    \graphicspath{{../Immagini/}}

    \author{Lorenzo Vaccarecci}
}
\preamble

\title{Progettazione Logica}
\date{18 Marzo 2024}

\begin{document}
\maketitle
\begin{center}
    \begin{tikzpicture}[node distance=2cm]
        \node[entity] (er) {Schema ER + vincoli};
        \node[entity] (carlavoro) [right of=er, xshift=3cm] {Carico di lavoro};
        \node[entity] (voldati) [right of=carlavoro, xshift=3cm] {Volume dei dati};
        \node[attribute] (ristrut) [below of=carlavoro] {Ristrutturazione};
        \node[attribute] (trad) [below of=ristrut] {Traduzione};
        \node[entity] (fine) [below of=trad] {Schema relazionale + vincoli};
        \draw[->] (er) -- (ristrut);
        \draw[->] (carlavoro) -- (ristrut);
        \draw[->] (voldati) -- (ristrut);
        \draw[->] (ristrut) -- (trad);
        \draw[->] (trad) -- (fine);
    \end{tikzpicture}
\end{center}
\textit{Volume dei dati: quante istruzioni per ogni entità/associazione}
\section{Ristrutturazione}
\subsection{Attributi composti}
\begin{enumerate}
    \item Se non utilizzato per ricerce, si può mettere tutto in un unico attributo.
    \item Se utilizzato per ricerce, si può mettere in attributi separati.
    \item Si trasforma in entità.
\end{enumerate}
\subsection{Attributi multivalore}
\begin{enumerate}
    \item Se abbiamo la cardinalità generica ($n$), si trasforma in relazione.
    \item Se abbiamo cardinalità limitata, si trasforma in attributi separati.
\end{enumerate}\newpage
\subsection{Gerarchie}
\begin{enumerate}
    \item eliminare il padre (solo se la gerarchia è totale): eliminiamo il padre e le associazioni si collegano ai figli, gli attributi del padre vengono passati ai figli
    \item eliminare i figli: eliminiamo i figli ma le associazioni si collegano al padre e  gli eventuali attributi vengono passati al padre come opzionali
    \item trasformare in associazione: si trasforma la gerarchia in associazione con cardinalità $(0,1)$ e $(1,1)$ con chiave esterna semplice (dal padre al figlio).
\end{enumerate}
\end{document}
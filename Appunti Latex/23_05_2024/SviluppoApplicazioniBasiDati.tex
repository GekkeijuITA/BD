\documentclass[12pt]{article}
\usepackage[utf8]{inputenc}
\newcommand\preamble{
    \usepackage[italian]{babel}
    \usepackage{geometry}
    \usepackage{amsmath}
    \usepackage{amssymb}
    \usepackage{graphicx}
    \usepackage{ulem}
    \usepackage[table, dvipsnames]{xcolor}
    \usepackage{tikz}
    \usepackage{qtree}
    \usepackage{spverbatim}
    \usepackage{listings}
    \usepackage{hyperref}

    \usetikzlibrary{er, positioning}
    \usetikzlibrary{arrows}

    \tikzset {multi attribute/.style={attribute,double distance=1.5pt}}
    \tikzset {derived attribute/.style={attribute,dashed}}
    \tikzset {total/.style={double distance=1.5pt}}
    %\tikzset {every entity/.style={ draw=orange,fill=orange!20}}
    %\tikzset {every attribute/.style={draw=purple,fill=purple!20}}
    %\tikzset {every relationship/.style={draw=green,fill=green!20}}

    \geometry{margin=2cm}
    \let\olditemize\itemize
    \renewcommand\itemize{\olditemize\setlength\itemsep{0em}}
    \graphicspath{{../Immagini/}}

    \author{Lorenzo Vaccarecci}

    \definecolor{codegreen}{rgb}{0,0.6,0}
    \definecolor{codegray}{rgb}{0.5,0.5,0.5}
    \definecolor{codepurple}{rgb}{0.58,0,0.82}
    \definecolor{backcolour}{rgb}{0.95,0.95,0.92}

    \lstdefinestyle{mystyle}{
        backgroundcolor=\color{backcolour},   
        commentstyle=\color{codegreen},
        keywordstyle=\color{magenta},
        numberstyle=\tiny\color{codegray},
        stringstyle=\color{codepurple},
        basicstyle=\ttfamily\footnotesize,
        breakatwhitespace=false,         
        breaklines=true,                 
        captionpos=b,                    
        keepspaces=true,                 
        numbers=left,                    
        numbersep=5pt,                  
        showspaces=false,                
        showstringspaces=false,
        showtabs=false,                  
        tabsize=2
    }

    \lstset{style=mystyle}

    \hypersetup{
        colorlinks=true,
        linkcolor=blue,
        filecolor=magenta,      
        urlcolor=blue,
        pdfpagemode=FullScreen,
    }

    \urlstyle{same}
}
\newcommand {\key}[1]{\underline{#1}}
\newcommand {\question}[1]{\textit{#1}\\}
\newcommand {\important}[1]{\textcolor{red}{#1}}
\newcommand {\remark}[1]{\textcolor{Cyan}{#1}}
\preamble

\title{Sviluppo di Applicazioni per Basi di Dati}
\date{23 Maggio 2024}

\begin{document}
\maketitle
\section{Problema}
SQL non permette di eseguire elaborazioni complesse sui dati e interagire con il sistema operativo. SQL quindi non è \remark{completo}:
\begin{itemize}
    \item \important{Computazionalmente}: banalmente mancano i costrutti di scelta o iterazione
    \item \important{Operazionalmente}: mancano i costrutti per interagire con il sistema operativo
\end{itemize}
\section{Soluzione}
Si \important{combina} SQL con un linguaggio di programmazione generico.\\
Approcci:
\begin{itemize}
    \item \important{interno}(al DBMS): lato server, il codice eseguito dal DBMS
    \item \important{esterno}(al DBMS): lato client o server, il codice eseguito da un'applicazione che invia la query al DBMS (\textit{es. un programma Java}).
\end{itemize}
\section{Routine}
I programmi sono organizzati in routine (procedure o funzioni) e possono essere eseguite direttamente dal DBMS o chiamate da applicazioni basate su accoppiamento esterno.
\subsection{Routine Anonime}
Sono dei blocchi di codice a cui non viene dato un nome:
\begin{lstlisting}
    DO
    $$
        <Corpo Routine>
    $$
    LANGUAGE <Linguaggio>;
\end{lstlisting}
\underline{Il linguaggio usato deve essere supportato dal DBMS}.
\section{Attività}
Questa parte di codice crea una routine anonima che stampa un messaggio:
\begin{lstlisting}[language=SQL]
    DO
    $$
        BEGIN
            RAISE NOTICE 'Hello, world!';
        END
    $$;
\end{lstlisting}
Per dichiarare delle variabili
\begin{lstlisting}[language=SQL]
    DO
    $$
        DECLARE
            <Nome Variabile> <Tipo Variabile> [:= <Valore> | DEFAULT <Valore>];
        BEGIN
            <Corpo Routine>
        END
    $$
\end{lstlisting}
La dichiarazione è molto simile a quella usata per creare le colonne di una tabella.\\
Per poter stampare le variabili:
\begin{lstlisting}[language=SQL]
    RAISE NOTICE 'Messaggio %', <Variabile>;
\end{lstlisting}
Se la variabile non è stata inizializzata, stamperà \texttt{NULL}. Ovviamente se una variabile ha l'opzione \texttt{NOT NULL} non possiamo assegnarli \texttt{NULL}.\\
A una variabile possiamo assegnare il risultato di una query:
\begin{lstlisting}[language=SQL]
    denominazioneCdL := (SELECT denominazione FROM corsidilaurea WHERE id = 9)
\end{lstlisting}
While loop:
\begin{lstlisting}[language=SQL]
    DO
    $$
        BEGIN
            WHILE <Condizione>
            LOOP
                <Corpo Loop>
            END LOOP;
        END
    $$
\end{lstlisting}
\subsection{Cursore}
Serve per manipolare il risultato di una query.
\begin{lstlisting}[language=SQL]
    DECLARE
        <Nome Cursore> CURSOR FOR <Query>;
    BEGIN
        OPEN <Nome Cursore>; -- Apre il cursore ed esegue la query
        FETCH <Nome Cursore> INTO <Variabile>; -- Prende il risultato della query e lo mette nella variabile
    END
\end{lstlisting}
\end{document}
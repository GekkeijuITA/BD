\documentclass[12pt]{article}
\usepackage[utf8]{inputenc}
\newcommand\preamble{
    \usepackage[italian]{babel}
    \usepackage{geometry}
    \usepackage{amsmath}
    \usepackage{amssymb}
    \usepackage{graphicx}
    \usepackage{ulem}
    \usepackage[dvipsnames]{xcolor}

    \geometry{margin=2cm}
    \let\olditemize\itemize
    \renewcommand\itemize{\olditemize\setlength\itemsep{0em}}
    \graphicspath{{../Immagini/}}

    \author{Lorenzo Vaccarecci}
}
\preamble

\title{Trigger}
\date{27 Maggio 2024}

\begin{document}
\maketitle
\section{Introduzione}
Un DBMS si dice \remark{attivo} quando reagisce ad eventi eseguendo operazioni definite dal progettista $\rightarrow$ ADBMS. 
\section{Trigger}
Un trigger è un insieme di istruzioni SQL che viene eseguito automaticamente dal DBMS in risposta ad un evento specifico che si verifica nel database.\\
\textbf{Benefici:}
\begin{itemize}
    \item Efficienza
    \item Costi di manutenzione
    \item Uniformità di gestione dei dati
    \item Integrazione con le altre componenti del DBMS
\end{itemize}
\subsection{Paradigma ECA(Evento-Condizione-Azione)}
\begin{lstlisting}
    ON evento
    IF condizione
    THEN azione
\end{lstlisting}
Un evento è qualcosa che accade e che può essere
mappato, dal punto di vista del sistema, in un istante di tempo (\textit{es. \texttt{INSERT,UPDATE,DELETE}}).
\subsection{In PostgreSQL}
\begin{lstlisting}[language=SQL]
    CREATE TRIGGER nome_trigger
    {BEFORE | AFTER | INSTEAD OF} {evento}
    ON nome_tabella
    [FOR EACH ROW]
    [WHEN ([NEW | OLD] condizione)]
    EXECUTE FUNCTION nome_funzione
\end{lstlisting}
\begin{itemize}
    \item \texttt{BEFORE}: il trigger viene eseguito prima dell'evento
    \item \texttt{AFTER}: il trigger viene eseguito dopo l'evento
    \item \texttt{INSTEAD OF}: il trigger viene eseguito al posto dell'evento (\textbf{solo per viste})
    \item \texttt{FOR EACH ROW}: il trigger viene eseguito per ogni riga coinvolta dall'evento
    \item \texttt{OLD}: rappresenta il valore della riga prima dell'evento
    \item \texttt{NEW}: rappresenta il valore della riga dopo l'evento
\end{itemize}
\end{document}
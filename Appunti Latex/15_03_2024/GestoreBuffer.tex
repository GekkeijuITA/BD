\documentclass[12pt]{article}
\usepackage[utf8]{inputenc}
\newcommand\preamble{
    \usepackage[italian]{babel}
    \usepackage{geometry}
    \usepackage{amsmath}
    \usepackage{amssymb}
    \usepackage{graphicx}
    \usepackage{ulem}
    \usepackage[table, dvipsnames]{xcolor}
    \usepackage{tikz}
    \usepackage{qtree}
    \usepackage{spverbatim}
    \usepackage{listings}
    \usepackage{hyperref}

    \usetikzlibrary{er, positioning}
    \usetikzlibrary{arrows}

    \tikzset {multi attribute/.style={attribute,double distance=1.5pt}}
    \tikzset {derived attribute/.style={attribute,dashed}}
    \tikzset {total/.style={double distance=1.5pt}}
    %\tikzset {every entity/.style={ draw=orange,fill=orange!20}}
    %\tikzset {every attribute/.style={draw=purple,fill=purple!20}}
    %\tikzset {every relationship/.style={draw=green,fill=green!20}}

    \geometry{margin=2cm}
    \let\olditemize\itemize
    \renewcommand\itemize{\olditemize\setlength\itemsep{0em}}
    \graphicspath{{../Immagini/}}

    \author{Lorenzo Vaccarecci}

    \definecolor{codegreen}{rgb}{0,0.6,0}
    \definecolor{codegray}{rgb}{0.5,0.5,0.5}
    \definecolor{codepurple}{rgb}{0.58,0,0.82}
    \definecolor{backcolour}{rgb}{0.95,0.95,0.92}

    \lstdefinestyle{mystyle}{
        backgroundcolor=\color{backcolour},   
        commentstyle=\color{codegreen},
        keywordstyle=\color{magenta},
        numberstyle=\tiny\color{codegray},
        stringstyle=\color{codepurple},
        basicstyle=\ttfamily\footnotesize,
        breakatwhitespace=false,         
        breaklines=true,                 
        captionpos=b,                    
        keepspaces=true,                 
        numbers=left,                    
        numbersep=5pt,                  
        showspaces=false,                
        showstringspaces=false,
        showtabs=false,                  
        tabsize=2
    }

    \lstset{style=mystyle}

    \hypersetup{
        colorlinks=true,
        linkcolor=blue,
        filecolor=magenta,      
        urlcolor=blue,
        pdfpagemode=FullScreen,
    }

    \urlstyle{same}
}
\newcommand {\key}[1]{\underline{#1}}
\newcommand {\question}[1]{\textit{#1}\\}
\newcommand {\important}[1]{\textcolor{red}{#1}}
\newcommand {\remark}[1]{\textcolor{Cyan}{#1}}
\preamble

\title{Gestore delle strutture di memorizzazione - Buffer}
\date{15 Marzo 2024}

\begin{document}
\maketitle
\section{Introduzione}
L'obiettivo è minimizzare il numero di accessi alla memoria non volatile, che è molto lenta. Il buffer viene gestito dal DBMS.\\
Mantenere più blocchi possibili in memoria principale in modo da evitare riletture da memoria non volatile.
\section{Il Buffer}
Il buffer è organizzato in pagine, che hanno la stessa dimensione delle pagine/blocchi su disco.
\includegraphics[width=\textwidth]{buffer.png}
\section{Gestione del buffer}
Dopo l'esecuzione di una interrogazione, il \textbf{Buffer Manager} (BM) controlla prima nel buffer e se la pagina non è presente:
\begin{itemize}
    \item Cerca una pagina nel buffer libera
    \item Se non è presente, cerca una pagina da sostituire
    \item Se la pagina da sostituire è stata modificata, la scrive su disco
    \item A questo punto la pagina è libera e può essere sovrascritta
\end{itemize}
Quando una pagina è presente nel buffer, le operazioni di lettura e scrittura possono essere effettuate su di essa.
\subsection{Politiche di sostituzione}
Il BM sceglie quale politica usare in base alle informazioni che ha.
\begin{description}
    \item[\textbf{LRU (Least Recently Used)}]: La pagina meno recentemente usata viene sostituita. 
    \item[\textbf{MRU (Most Recently Used)}]: La pagina più recentemente usata viene sostituita. Spesso viene usata questa politica perchè il sistema sa che dovrà rileggere questo blocco ma non nell'immediato futuro.
\end{description}
\end{document}
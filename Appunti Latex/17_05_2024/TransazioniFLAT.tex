\documentclass[12pt]{article}
\usepackage[utf8]{inputenc}
\newcommand\preamble{
    \usepackage[italian]{babel}
    \usepackage{geometry}
    \usepackage{amsmath}
    \usepackage{amssymb}
    \usepackage{graphicx}
    \usepackage{ulem}
    \usepackage[table, dvipsnames]{xcolor}
    \usepackage{tikz}
    \usepackage{qtree}
    \usepackage{spverbatim}
    \usepackage{listings}
    \usepackage{hyperref}

    \usetikzlibrary{er, positioning}
    \usetikzlibrary{arrows}

    \tikzset {multi attribute/.style={attribute,double distance=1.5pt}}
    \tikzset {derived attribute/.style={attribute,dashed}}
    \tikzset {total/.style={double distance=1.5pt}}
    %\tikzset {every entity/.style={ draw=orange,fill=orange!20}}
    %\tikzset {every attribute/.style={draw=purple,fill=purple!20}}
    %\tikzset {every relationship/.style={draw=green,fill=green!20}}

    \geometry{margin=2cm}
    \let\olditemize\itemize
    \renewcommand\itemize{\olditemize\setlength\itemsep{0em}}
    \graphicspath{{../Immagini/}}

    \author{Lorenzo Vaccarecci}

    \definecolor{codegreen}{rgb}{0,0.6,0}
    \definecolor{codegray}{rgb}{0.5,0.5,0.5}
    \definecolor{codepurple}{rgb}{0.58,0,0.82}
    \definecolor{backcolour}{rgb}{0.95,0.95,0.92}

    \lstdefinestyle{mystyle}{
        backgroundcolor=\color{backcolour},   
        commentstyle=\color{codegreen},
        keywordstyle=\color{magenta},
        numberstyle=\tiny\color{codegray},
        stringstyle=\color{codepurple},
        basicstyle=\ttfamily\footnotesize,
        breakatwhitespace=false,         
        breaklines=true,                 
        captionpos=b,                    
        keepspaces=true,                 
        numbers=left,                    
        numbersep=5pt,                  
        showspaces=false,                
        showstringspaces=false,
        showtabs=false,                  
        tabsize=2
    }

    \lstset{style=mystyle}

    \hypersetup{
        colorlinks=true,
        linkcolor=blue,
        filecolor=magenta,      
        urlcolor=blue,
        pdfpagemode=FullScreen,
    }

    \urlstyle{same}
}
\newcommand {\key}[1]{\underline{#1}}
\newcommand {\question}[1]{\textit{#1}\\}
\newcommand {\important}[1]{\textcolor{red}{#1}}
\newcommand {\remark}[1]{\textcolor{Cyan}{#1}}
\preamble

\title{Transazioni (FLAT)}
\date{17 Maggio 2024}

\begin{document}
\maketitle
\section{Cosa è una transazione}
\important{E' un'unità di elaborazione} che corrisponde a \remark{una serie di operazioni fisiche elementari} sulla base di dati a cui viene garantita un'esecuzione che soddisfa alcune buone proprietà.
\section{Proprietà ACID}
\begin{itemize}
    \item \important{A}tomicità: se succede qualcosa di sbagliato allora il sistema torna allo stato precedente alla transazione.
    \item \important{C}onsistenza: lo stato iniziale e finale di una transazione devono sempre soddisfare tutti i vincoli di integrita' esistenti. \textbf{Gli stati intermedi possono pero' violare la consistenza.} 
    \item \important{I}solamento: se si hanno più transazioni contemporaneamente, "non si pestano i piedi".
    \item \important{D}urabilità (persistenza): nel momento in cui la transazione è completata, le modifiche sono permanenti. 
\end{itemize}
Le proprietà ACID sono garantite da specifiche componenti presenti del DBMS (\important{Transaction Manager}).
\section{Comandi SQL}
In SQL si usano i comandi \texttt{BEGIN} e \texttt{COMMIT}:
\begin{lstlisting}[language=SQL]
    BEGIN;
        ...
    COMMIT;
\end{lstlisting}
Il comando \texttt{COMMIT} restituisce \texttt{COMMIT} se è andato a buon fine, altrimenti restituisce \texttt{ROLLBACK}.\\
Un altro modo è:
\begin{lstlisting}[language=SQL]
    BEGIN;
        ...
    ROLLBACK;
\end{lstlisting}
In questo modo si annullano tutte le operazioni come se ci fosse stato un errore (anche se tutto corretto).
\end{document}
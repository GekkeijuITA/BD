\documentclass[12pt]{article}
\usepackage[italian]{babel}
\usepackage{geometry}
\usepackage{amsmath}
\usepackage{amssymb}
\usepackage{graphicx}
\usepackage{ulem}
\usepackage{array}

\geometry{margin=2cm}
\newtheorem{definition}{Definizione}
\newtheorem{example}{Esempio}

\title{DBMS}
\author{Lorenzo Vaccarecci}
\date{4 Marzo 2024}

\graphicspath{{./Immagini/}}

\begin{document}
\maketitle
\section{Introduzione}
\begin{description}
    \item[Base di dati]: collezione di dati tra loro correlati, che rappresentano le informazioni di interesse in un sistema informativo.
    \item[Sistema di gestione di basi di dati (DBMS)]: sistema software che fornisce gli strumenti necessari a gestire basi di dati. 
\end{description}
\subsection{Differenze tra file system e DBMS}
\begin{center}
    \begin{tabular}{| m{6em} | m{20em} | m{15em} |}
        \hline
         & \textbf{FS/SO} & \textbf{DBMS} \\
        \hline
        \textbf{Ridondanza e inconsistenza} & 
        Se ho un nome in file diversi, ad esempio, e lo voglio modificare allora devo modificarlo in tutti i file in cui è presente & 
        Modello per descrivere le mie \uline{entità} e le associazioni tra le entità \\
        \hline
        \textbf{Difficoltà di accesso ai dati} &
        Per ogni richiesta dovrei avere un nuovo programma &
        \begin{itemize}
            \setlength\itemsep{0em}
            \item Disponibili linguaggi che permettono di specificare in modo semplice le richieste sui nostri dati
            \item Uso interattivo
        \end{itemize}
        \\
        \hline
        \textbf{Integrità dei dati} &
        Il vincolo di integrità deve essere preso in considerazione da ogni programma che utilizza dati corrispondenti&
        \begin{itemize}
            \setlength\itemsep{0em}
            \item I vincoli di integrità vengono specificati nel sistema
            \item Il sistema li verificherà ogni volta che i dati vengono modificati
        \end{itemize} \\
        \hline
        \textbf{Protezione dei dati} &
        A livello di File&
        \begin{itemize}
            \setlength\itemsep{0em}
            \item Accesso concorrente e protezione dei dati a granularità più fine (in base al conenuto di interesse)
        \end{itemize} \\
        \hline
    \end{tabular}
\end{center}
\subsection{Carte vincenti dei DBMS}
\begin{itemize}
    \item \textbf{Schema vs Istanza}: Lo schema è la struttura "della tabella" che include i vincoli di integrità; l'istanza è il contenuto dello schema.
    \item \textbf{Linguaggi \uline{dichiarativi}}: dico cosa voglio fare, non come farlo. L'algoritmo viene scelto dal DBMS.
\end{itemize}
\end{document}
\documentclass[12pt]{article}
\usepackage[utf8]{inputenc}
\newcommand\preamble{
    \usepackage[italian]{babel}
    \usepackage{geometry}
    \usepackage{amsmath}
    \usepackage{amssymb}
    \usepackage{graphicx}
    \usepackage{ulem}
    \usepackage[dvipsnames]{xcolor}

    \geometry{margin=2cm}
    \let\olditemize\itemize
    \renewcommand\itemize{\olditemize\setlength\itemsep{0em}}
    \graphicspath{{../Immagini/}}

    \author{Lorenzo Vaccarecci}
}
\preamble

\title{Elaboratore delle interrogazioni}
\date{12 Aprile 2024}
\begin{document}
\maketitle
\section{Introduzione}
Consideriamo la seguente interrogazione SQL
\begin{verbatim}
    SELECT B,D
    FROM R,S
    WHERE R.A = "c" AND S.E = 2 AND R.C = S.C
\end{verbatim}
Può essere riscritta in algebra relazionale(linguaggio operazionale) come\\
\(
  \Pi_{B,D}(\sigma_{R.A="c" \land S.E=2 \land R.C=S.C}(R \times S))  
\)
\\L'espressione algebrica rappresenta un algoritmo (logico) di esecuzione che opera su tabelle e si può rappresentare come un albero.\\
Piano di esecuzione logico (bottom-up):
\begin{center}
    \Tree[ .$\Pi_{B,D}$ [ .$\sigma_{R.A="c" \land S.E=2 \land R.C=S.C}$ [ .$\times$ [ .$R$ ] [ .$S$ ] ] ] ]
\end{center}
Altra possibile strategia corrisponde a piano logico alternativo:
[Appunti mancanti (pag.11-12 circa delle slide)]
\begin{center}
    % Albero corrispondente al piano II
\end{center}
Esecuzione più efficiente?\\
\textcolor{red}{il piano II, in quanto evita l'esecuzione del prodotto cartesiano riducendo la dimensione dei risultati intermedi generati e il numero di operazioni eseguite (anticipa le select).}
\section{Dal piano logico al piano fisico}
Esistono molteplici strategie logiche e molteplici algoritmi.
\textcolor{red}{Il compito del query processor è individuare il piano fisico di esecuzione più efficiente, a partire da uno schema logico e uno schema fisico in input.} Il costo di determinare la strategia ottima può essere elevato, il vantaggio in termini di tempo di esecuzione che se ne ricava è tuttavia tale da rendere preferibile eseguire l'ottimizzazione.
\subsection{Passi principali}
\begin{itemize}
    \item compilatore
    \item ottimizzatore (ottimizza a livello logico e fisico)
    \item query engine (esegue il piano fisico)
\end{itemize}
\subsubsection{Esempio}
\begin{verbatim}
    SELECT B,D
    FROM R,S
    WHERE R.A = "c" AND S.E = 2 AND R.C = S.C
\end{verbatim}
\begin{enumerate}
    \item parser
    \item traduzione (FROM, WHERE, SELECT) \(\rightarrow
        \Pi_{B,D}(\sigma_{R.A="c" \land S.E=2 \land R.C=S.C}(R \times S))  
      \)
    \item ottimizzazione 
    \item esecuzione
\end{enumerate}
\end{document}
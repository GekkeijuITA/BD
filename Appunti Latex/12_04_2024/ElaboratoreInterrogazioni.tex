\documentclass[12pt]{article}
\usepackage[utf8]{inputenc}
\newcommand\preamble{
    \usepackage[italian]{babel}
    \usepackage{geometry}
    \usepackage{amsmath}
    \usepackage{amssymb}
    \usepackage{graphicx}
    \usepackage{ulem}
    \usepackage[table, dvipsnames]{xcolor}
    \usepackage{tikz}
    \usepackage{qtree}
    \usepackage{spverbatim}
    \usepackage{listings}
    \usepackage{hyperref}

    \usetikzlibrary{er, positioning}
    \usetikzlibrary{arrows}

    \tikzset {multi attribute/.style={attribute,double distance=1.5pt}}
    \tikzset {derived attribute/.style={attribute,dashed}}
    \tikzset {total/.style={double distance=1.5pt}}
    %\tikzset {every entity/.style={ draw=orange,fill=orange!20}}
    %\tikzset {every attribute/.style={draw=purple,fill=purple!20}}
    %\tikzset {every relationship/.style={draw=green,fill=green!20}}

    \geometry{margin=2cm}
    \let\olditemize\itemize
    \renewcommand\itemize{\olditemize\setlength\itemsep{0em}}
    \graphicspath{{../Immagini/}}

    \author{Lorenzo Vaccarecci}

    \definecolor{codegreen}{rgb}{0,0.6,0}
    \definecolor{codegray}{rgb}{0.5,0.5,0.5}
    \definecolor{codepurple}{rgb}{0.58,0,0.82}
    \definecolor{backcolour}{rgb}{0.95,0.95,0.92}

    \lstdefinestyle{mystyle}{
        backgroundcolor=\color{backcolour},   
        commentstyle=\color{codegreen},
        keywordstyle=\color{magenta},
        numberstyle=\tiny\color{codegray},
        stringstyle=\color{codepurple},
        basicstyle=\ttfamily\footnotesize,
        breakatwhitespace=false,         
        breaklines=true,                 
        captionpos=b,                    
        keepspaces=true,                 
        numbers=left,                    
        numbersep=5pt,                  
        showspaces=false,                
        showstringspaces=false,
        showtabs=false,                  
        tabsize=2
    }

    \lstset{style=mystyle}

    \hypersetup{
        colorlinks=true,
        linkcolor=blue,
        filecolor=magenta,      
        urlcolor=blue,
        pdfpagemode=FullScreen,
    }

    \urlstyle{same}
}
\newcommand {\key}[1]{\underline{#1}}
\newcommand {\question}[1]{\textit{#1}\\}
\newcommand {\important}[1]{\textcolor{red}{#1}}
\newcommand {\remark}[1]{\textcolor{Cyan}{#1}}
\preamble

\title{Elaboratore delle interrogazioni}
\date{12 Aprile 2024}
\begin{document}
\maketitle
\section{Introduzione}
Consideriamo la seguente interrogazione SQL
\begin{verbatim}
    SELECT B,D
    FROM R,S
    WHERE R.A = "c" AND S.E = 2 AND R.C = S.C
\end{verbatim}
Può essere riscritta in algebra relazionale(linguaggio operazionale) come\\
\(
  \Pi_{B,D}(\sigma_{R.A="c" \land S.E=2 \land R.C=S.C}(R \times S))  
\)
\\L'espressione algebrica rappresenta un algoritmo (logico) di esecuzione che opera su tabelle e si può rappresentare come un albero.\\
Piano di esecuzione logico (bottom-up):
\begin{center}
    \Tree[ .$\Pi_{B,D}$ [ .$\sigma_{R.A="c" \land S.E=2 \land R.C=S.C}$ [ .$\times$ [ .$R$ ] [ .$S$ ] ] ] ]
\end{center}
Altra possibile strategia corrisponde a piano logico alternativo:
[Appunti mancanti (pag.11-12 circa delle slide)]
\begin{center}
    % Albero corrispondente al piano II
\end{center}
Esecuzione più efficiente?\\
\textcolor{red}{il piano II, in quanto evita l'esecuzione del prodotto cartesiano riducendo la dimensione dei risultati intermedi generati e il numero di operazioni eseguite (anticipa le select).}
\section{Dal piano logico al piano fisico}
Esistono molteplici strategie logiche e molteplici algoritmi.
\textcolor{red}{Il compito del query processor è individuare il piano fisico di esecuzione più efficiente, a partire da uno schema logico e uno schema fisico in input.} Il costo di determinare la strategia ottima può essere elevato, il vantaggio in termini di tempo di esecuzione che se ne ricava è tuttavia tale da rendere preferibile eseguire l'ottimizzazione.
\subsection{Passi principali}
\begin{itemize}
    \item compilatore
    \item ottimizzatore (ottimizza a livello logico e fisico)
    \item query engine (esegue il piano fisico)
\end{itemize}
\subsubsection{Esempio}
\begin{verbatim}
    SELECT B,D
    FROM R,S
    WHERE R.A = "c" AND S.E = 2 AND R.C = S.C
\end{verbatim}
\begin{enumerate}
    \item parser
    \item traduzione (FROM, WHERE, SELECT) \(\rightarrow
        \Pi_{B,D}(\sigma_{R.A="c" \land S.E=2 \land R.C=S.C}(R \times S))  
      \)
    \item ottimizzazione 
    \item esecuzione
\end{enumerate}
\end{document}
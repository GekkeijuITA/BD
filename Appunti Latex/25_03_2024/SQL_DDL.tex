\documentclass[12pt]{report}
\usepackage[utf8]{inputenc}
\newcommand\preamble{
    \usepackage[italian]{babel}
    \usepackage{geometry}
    \usepackage{amsmath}
    \usepackage{amssymb}
    \usepackage{graphicx}
    \usepackage{ulem}
    \usepackage[table, dvipsnames]{xcolor}
    \usepackage{tikz}
    \usepackage{qtree}
    \usepackage{spverbatim}
    \usepackage{listings}
    \usepackage{hyperref}

    \usetikzlibrary{er, positioning}
    \usetikzlibrary{arrows}

    \tikzset {multi attribute/.style={attribute,double distance=1.5pt}}
    \tikzset {derived attribute/.style={attribute,dashed}}
    \tikzset {total/.style={double distance=1.5pt}}
    %\tikzset {every entity/.style={ draw=orange,fill=orange!20}}
    %\tikzset {every attribute/.style={draw=purple,fill=purple!20}}
    %\tikzset {every relationship/.style={draw=green,fill=green!20}}

    \geometry{margin=2cm}
    \let\olditemize\itemize
    \renewcommand\itemize{\olditemize\setlength\itemsep{0em}}
    \graphicspath{{../Immagini/}}

    \author{Lorenzo Vaccarecci}

    \definecolor{codegreen}{rgb}{0,0.6,0}
    \definecolor{codegray}{rgb}{0.5,0.5,0.5}
    \definecolor{codepurple}{rgb}{0.58,0,0.82}
    \definecolor{backcolour}{rgb}{0.95,0.95,0.92}

    \lstdefinestyle{mystyle}{
        backgroundcolor=\color{backcolour},   
        commentstyle=\color{codegreen},
        keywordstyle=\color{magenta},
        numberstyle=\tiny\color{codegray},
        stringstyle=\color{codepurple},
        basicstyle=\ttfamily\footnotesize,
        breakatwhitespace=false,         
        breaklines=true,                 
        captionpos=b,                    
        keepspaces=true,                 
        numbers=left,                    
        numbersep=5pt,                  
        showspaces=false,                
        showstringspaces=false,
        showtabs=false,                  
        tabsize=2
    }

    \lstset{style=mystyle}

    \hypersetup{
        colorlinks=true,
        linkcolor=blue,
        filecolor=magenta,      
        urlcolor=blue,
        pdfpagemode=FullScreen,
    }

    \urlstyle{same}
}
\newcommand {\key}[1]{\underline{#1}}
\newcommand {\question}[1]{\textit{#1}\\}
\newcommand {\important}[1]{\textcolor{red}{#1}}
\newcommand {\remark}[1]{\textcolor{Cyan}{#1}}
\preamble

\title{SQL (DDL)}
\date{25 Marzo 2024}
\begin{document}
\maketitle
\tableofcontents
\chapter{Comandi}
\section{Creare tabelle}
\begin{spverbatim}
    CREATE TABLE *nome* (...);
\end{spverbatim}
\section{Inserire tuple}
\begin{spverbatim}
    INSERT INTO *Tabella* VALUES (...);
\end{spverbatim}
\section{Modificare attributi}
\begin{spverbatim}
    UPDATE *Tabella* SET *attributo* TO *valore*;
\end{spverbatim}
\section{Cancellare}
\begin{spverbatim}
    DELETE FROM *Tabella*; % Cancella tutto

    % Cancella le tuple che soddisfano la condizione
    DELETE FROM *Tabella* WHERE *condizione*;
\end{spverbatim}
\section{Ricerca}
\begin{spverbatim}
    WHERE *condizione*;
\end{spverbatim}
\chapter{Tipi}
\section{Decimal}
Prende come argomento il numero di cifre ad esempio:
\begin{spverbatim}
    *attributo* DECIMAL(4) % Andrà da -9999 a 9999
\end{spverbatim}
\section{Varchar}
Lo usiamo solo se abbiamo un limite massimo \textbf{e} non sappiamo se ci saranno delle stringhe più corte
\begin{spverbatim}
    *attributo* VARCHAR(numeroCaratteri)
\end{spverbatim}
\section{Char}
Lo usiamo nel caso in cui sappiamo che la stringa sarà, per la maggior parte delle volte, esattamente quel numero
\begin{spverbatim}
    *attributo* CHAR(numeroCaratteri);
\end{spverbatim}
\section{Date}
Lo usiamo per le date
\begin{spverbatim}
    *attributo* DATE
\end{spverbatim}
\subsection{Opzioni}
\begin{spverbatim}
    CURRENT_DATE % Restituisce la data corrente (anno/mese/giorno)
\end{spverbatim}
\section{Numeric}
Lo usiamo per i "float" indicando il numero di cifre prima e dopo la virgola
\begin{spverbatim}
    *attributo* NUMERIC(#int,#dec)
\end{spverbatim}
ad esempio
\begin{spverbatim}
    *attributo* NUMERIC(3,2) % -999.99 a 999.99
\end{spverbatim}
\chapter{Opzioni attributi}
\section{NOT NULL}
Serve per avere un attributo obbligatorio
\begin{spverbatim}
    *attributo* *TIPO* NOT NULL
\end{spverbatim}
\section{DEFAULT}
Serve per avere un valore predefinito in un attributo
\begin{spverbatim}
    *attributo* *TIPO* DEFAULT *valore*
\end{spverbatim}
\section{UNIQUE}
Serve per avere un attributo unico ma non chiave
\begin{spverbatim}
    *attributo* *TIPO* UNIQUE
\end{spverbatim}
oppure
\begin{spverbatim}
    *attributo* *TIPO*,
    UNIQUE(*attributo*)
\end{spverbatim}
\chapter{Chiavi}
\section{Chiave primaria}
\begin{spverbatim}
    *attributo* *TIPO* PRIMARY KEY
\end{spverbatim}
oppure (solo con attributo dichiarato in precedenza):
\begin{spverbatim}
    *attributo* *TIPO*,
    PRIMARY KEY(*attributo*)
\end{spverbatim}
\section{Chiave esterna}
    FOREIGN KEY (*chiave*,...) REFERENCES *tabella*
\subsection{Opzioni}
\subsubsection{On Delete}
    Per eliminare tutte le tuple della tabella esterna associate alla tupla corrente
    \begin{spverbatim}
        *Chiave esterna* ON DELETE CASCADE
    \end{spverbatim}
    Per impedire la cancellazione della tupla corrente se ha associate delle tuple in altre tabelle
    \begin{spverbatim}
        *Chiave esterna* ON DELETE RESTRICT
    \end{spverbatim}
\subsubsection{On Update Cascade}
\begin{spverbatim}
    *Chiave esterna* ON UPDATE CASCADE
\end{spverbatim}
\end{document}
\documentclass[12pt]{article}
\usepackage[utf8]{inputenc}
\newcommand\preamble{
    \usepackage[italian]{babel}
    \usepackage{geometry}
    \usepackage{amsmath}
    \usepackage{amssymb}
    \usepackage{graphicx}
    \usepackage{ulem}
    \usepackage[table, dvipsnames]{xcolor}
    \usepackage{tikz}
    \usepackage{qtree}
    \usepackage{spverbatim}
    \usepackage{listings}
    \usepackage{hyperref}

    \usetikzlibrary{er, positioning}
    \usetikzlibrary{arrows}

    \tikzset {multi attribute/.style={attribute,double distance=1.5pt}}
    \tikzset {derived attribute/.style={attribute,dashed}}
    \tikzset {total/.style={double distance=1.5pt}}
    %\tikzset {every entity/.style={ draw=orange,fill=orange!20}}
    %\tikzset {every attribute/.style={draw=purple,fill=purple!20}}
    %\tikzset {every relationship/.style={draw=green,fill=green!20}}

    \geometry{margin=2cm}
    \let\olditemize\itemize
    \renewcommand\itemize{\olditemize\setlength\itemsep{0em}}
    \graphicspath{{../Immagini/}}

    \author{Lorenzo Vaccarecci}

    \definecolor{codegreen}{rgb}{0,0.6,0}
    \definecolor{codegray}{rgb}{0.5,0.5,0.5}
    \definecolor{codepurple}{rgb}{0.58,0,0.82}
    \definecolor{backcolour}{rgb}{0.95,0.95,0.92}

    \lstdefinestyle{mystyle}{
        backgroundcolor=\color{backcolour},   
        commentstyle=\color{codegreen},
        keywordstyle=\color{magenta},
        numberstyle=\tiny\color{codegray},
        stringstyle=\color{codepurple},
        basicstyle=\ttfamily\footnotesize,
        breakatwhitespace=false,         
        breaklines=true,                 
        captionpos=b,                    
        keepspaces=true,                 
        numbers=left,                    
        numbersep=5pt,                  
        showspaces=false,                
        showstringspaces=false,
        showtabs=false,                  
        tabsize=2
    }

    \lstset{style=mystyle}

    \hypersetup{
        colorlinks=true,
        linkcolor=blue,
        filecolor=magenta,      
        urlcolor=blue,
        pdfpagemode=FullScreen,
    }

    \urlstyle{same}
}
\newcommand {\key}[1]{\underline{#1}}
\newcommand {\question}[1]{\textit{#1}\\}
\newcommand {\important}[1]{\textcolor{red}{#1}}
\newcommand {\remark}[1]{\textcolor{Cyan}{#1}}
\preamble

\title{Indici Hash}
\date{5 Aprile 2024}

\begin{document}
\maketitle
L’uso di indici ad albero ha lo svantaggio di richiedere la scansione di una struttura dati, memorizzata su disco, per localizzare i dati. Questo perché le associazioni $(k_{i}, r_{i})$ vengono mantenute in forma esplicita, come record in un file Gli indici hash al contrario mantengono le associazioni $(k_{i}, r_{i})$ in modo implicito, tramite l’uso di una funzione hash, definita sul dominio della chiave di ricerca.
\section{Caratteristiche Generali}
\begin{itemize}
    \item Una funzione hash su $K$ è una funzione \textcolor{red}{surgettiva $H(D_{K})\rightarrow\{0,\dots,M-1\}$}
    \item $M$ costante
\end{itemize}
\begin{center}
    \includegraphics[scale=0.5]{indicihashgen.png}
\end{center}
\textit{A ogni valore della funzione hash corrisponde un indirizzo in area primaria.}\newpage
\subsection*{Esempio}
$I_{A}(R) \quad H(D_{A}\rightarrow\{0,\ldots,2\})$, la funzione $H$ restituisce il numero del bucket.
\begin{itemize}
    \item $H(1)=1$
    \item $H(2)=2$
    \item $H(3)=0$
    \item $H(5)=2$
    \item $H(10)=1$
\end{itemize}
1. creo bucket index: 
\begin{tabular}{|c|c|c|}
    \hline
    0 & 1 & 2 \\
    \hline
\end{tabular}\\
\begin{tabular}{|c|c|}
    \hline
    A & B\\
    \hline
    1 & a \\
    2 & a \\
    3 & b \\
    5 & c \\
    10 & d \\
    3 & f \\
    \hline
\end{tabular} $\Rightarrow$
\begin{tabular}{|cc|}
    \hline
    \textbf{Bucket 0} & \hphantom{}\\
    3 & b\\
    3 & f \\
    \hline
    \textbf{Bucket 1} & \hphantom{}\\
    1 & a \\
    10 & d \\
    \hline
    \textbf{Bucket 2} & \hphantom{}\\
    2 & a \\
    5 & c \\
    \hline
\end{tabular}
\section{Ricerca per uguaglianza}
\begin{center}
    \includegraphics[scale=0.5]{indicihashuguaglianza.png}
\end{center}
\textit{Non supporta la ricerca per intervallo in quanto dovrei conoscere tutti i valori dell'intervallo in quanto la funzione hash non mantiene l'ordine.}
\section{Inserimento e Trabocchi}
\begin{center}
    \includegraphics[scale=0.5]{inidicihashinserimento.png}
\end{center}
Se il bucket è pieno, si alloca un nuovo blocco (\textit{trabocco}) dove c'è spazio chiamato \textcolor{red}{area di overflow} e se anche l'area di overflow è piena se ne crea un'altra e così via.
\subsection{Ricerca}
\begin{center}
    \includegraphics[scale=0.4]{indicihashricercatrabocco.png}
\end{center}
\end{document}
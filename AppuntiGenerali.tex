\documentclass[12pt, a4paper]{report}
\usepackage[utf8]{inputenc}
\newcommand\preamble{
    \usepackage[italian]{babel}
    \usepackage{geometry}
    \usepackage{amsmath}
    \usepackage{amssymb}
    \usepackage{graphicx}
    \usepackage{ulem}
    \usepackage[table, dvipsnames]{xcolor}
    \usepackage{tikz}
    \usepackage{qtree}
    \usepackage{spverbatim}
    \usepackage{listings}
    \usepackage{hyperref}

    \usetikzlibrary{er, positioning}
    \usetikzlibrary{arrows}

    \tikzset {multi attribute/.style={attribute,double distance=1.5pt}}
    \tikzset {derived attribute/.style={attribute,dashed}}
    \tikzset {total/.style={double distance=1.5pt}}
    %\tikzset {every entity/.style={ draw=orange,fill=orange!20}}
    %\tikzset {every attribute/.style={draw=purple,fill=purple!20}}
    %\tikzset {every relationship/.style={draw=green,fill=green!20}}

    \geometry{margin=2cm}
    \let\olditemize\itemize
    \renewcommand\itemize{\olditemize\setlength\itemsep{0em}}
    \graphicspath{{../Immagini/}}

    \author{Lorenzo Vaccarecci}

    \definecolor{codegreen}{rgb}{0,0.6,0}
    \definecolor{codegray}{rgb}{0.5,0.5,0.5}
    \definecolor{codepurple}{rgb}{0.58,0,0.82}
    \definecolor{backcolour}{rgb}{0.95,0.95,0.92}

    \lstdefinestyle{mystyle}{
        backgroundcolor=\color{backcolour},   
        commentstyle=\color{codegreen},
        keywordstyle=\color{magenta},
        numberstyle=\tiny\color{codegray},
        stringstyle=\color{codepurple},
        basicstyle=\ttfamily\footnotesize,
        breakatwhitespace=false,         
        breaklines=true,                 
        captionpos=b,                    
        keepspaces=true,                 
        numbers=left,                    
        numbersep=5pt,                  
        showspaces=false,                
        showstringspaces=false,
        showtabs=false,                  
        tabsize=2
    }

    \lstset{style=mystyle}

    \hypersetup{
        colorlinks=true,
        linkcolor=blue,
        filecolor=magenta,      
        urlcolor=blue,
        pdfpagemode=FullScreen,
    }

    \urlstyle{same}
}
\newcommand {\key}[1]{\underline{#1}}
\newcommand {\question}[1]{\textit{#1}\\}
\newcommand {\important}[1]{\textcolor{red}{#1}}
\newcommand {\remark}[1]{\textcolor{Cyan}{#1}}
\preamble

\usepackage{lipsum} 
\usepackage{caption}
\usepackage{fancyhdr}
\usepackage{setspace}
\usepackage{titlesec}

\geometry{top=1in, bottom=1in, left=1in, right=1in}

\pagestyle{fancy}
\fancyhf{}
\rfoot{\thepage}
\renewcommand{\headrulewidth}{0pt}

\titleformat{\chapter}[display]
  {\normalfont\fontsize{14}{16}\selectfont\bfseries\centering}{\MakeUppercase{\chaptertitlename}\ \thechapter}{20pt}{\fontsize{14}{16}\selectfont\bfseries}
\titleformat{\section}
  {\normalfont\fontsize{12}{14}\selectfont\bfseries}{\thesection}{1em}{}
\titleformat{\subsection}
  {\normalfont\fontsize{12}{14}\selectfont\bfseries}{\thesubsection}{1em}{}
\titleformat{\subsubsection}
  {\normalfont\fontsize{12}{14}\selectfont\bfseries}{\thesubsubsection}{1em}{}

  \makeatletter
  \renewcommand*\l@chapter[2]{%
    \ifnum \c@tocdepth >\m@ne
      \addpenalty{-\@highpenalty}%
      \vskip 1.0em \@plus\p@
      \setlength\@tempdima{1.5em}%
      \begingroup
        \parindent \z@ \rightskip \@pnumwidth
        \parfillskip -\@pnumwidth
        \leavevmode \bfseries
        \advance\leftskip\@tempdima
        \hskip -\leftskip
        #1\nobreak\hfil \nobreak\hb@xt@\@pnumwidth{\hss #2}\par
        \penalty\@highpenalty
      \endgroup
    \fi}
  
  \renewcommand*\l@section{\@dottedtocline{1}{0em}{2.3em}}
  \renewcommand*\l@subsection{\@dottedtocline{2}{2.3em}{3.2em}}
  \renewcommand*\l@subsubsection{\@dottedtocline{3}{5.5em}{4.1em}}
  \renewcommand*\l@paragraph{\@dottedtocline{4}{8.6em}{5em}}
  \renewcommand*\l@subparagraph{\@dottedtocline{5}{12em}{6em}}
  \makeatother 

\begin{document}
    \begin{titlepage}
        \centering
        % Title
        {\Huge \bfseries{[Appunti Basi di Dati]}\par}
        \vspace{1cm}
        {\large Appunti del corso "Basi di Dati" dell'Università degli Studi di Genova\par}  % Uncomment this for MCA
        % {\Large\bfseries Master of Science\par}   % Uncomment this for M.sc
        \vspace{1cm}
        {\large da\par}
        \vspace{2cm}
        {\Large\bfseries Lorenzo Vaccarecci\par}
        \vspace{1cm}
        % University Logo
        \includegraphics[width=0.7\textwidth]{Appunti Latex/Immagini/logoUnige.png}\\
        \vspace{1cm}
        % Department and University
        {\Large\bfseries Dipartimento di Informatica\par}
        {\Large\bfseries Università degli Studi di Genova\par}
        {\Large\bfseries 2024\par}
    \end{titlepage}
    \tableofcontents
    \chapter{Modello Relazionale}
        \section{Introduzione}
            Le interrogazioni sulle relazioni possono essere espresse in due formalismi:
            \begin{itemize}
                \item \textbf{Algebra relazionale}: le interrogazioni vengono espresse usando operatori specifici alle relazioni.
                \item \textbf{Calcolo relazionale}: le interrogazioni vengono espresse usando formule logiche.
            \end{itemize}
        \section{Relazioni}
            Un dominio è un insieme (anche infinito) di valori. Indicheremo con $\mathcal{D}$ l'insieme di tutti i domini.
            \subsection{Definizione: Prodotto cartesiano}
                Siano $D_{1},D_{2},\ldots,D_{k} \in \mathcal{D} \text{ con } k$ domini. Il prodotto cartesiano indicato con $D_{1}\times D_{2}\times \ldots \times D_{k}$, è definito come l'insieme $\left\{\left(v_{1},v_{2},\ldots,v_{k}\right)|v_{1}\in D_{1},\ldots,v_{k}\in D_{k}\right\}$.\\
                Gli elementi appartenenti al prodotto cartesiano sono detti \important{tuple}. Il prodotto cartesiano ha \textbf{grado} $k$.
            \subsection{Definizione: Relazione}
                Una relazione di $k$ domini è un sottoinsieme finito del prodotto cartesiano, ha \textbf{grado} $k$ quindi ogni tupla ha $k$ componenti. La \textbf{cardinalità} di una relazione è il numero di tuple appartenenti alla relazione. Una relazione è \underline{sempre} un insieme finito. \remark{Non vi sono tuple duplicate}.\\
            La coppia (nome di attributo, dominio) è detta \important{attributo}. L'uso di attributi permette di denotare le componenti di ogni tupla per nome piuttosto che per posizione.
            \subsection{Definizione: Schema di relazione}
                \begin{itemize}
                    \item $R$ un nome di relazione
                    \item $\left\{A_{1},A_{2},\ldots,A_{n}\right\}$ un insieme di nomi di attributi
                    \item $dom:\left\{A_{1},A_{2},\ldots,A_{n}\right\}\rightarrow \mathcal{D}$ una funzione totale che associa ad ogni nome di attributo il corrispondente dominio.
                \end{itemize}
                La coppia $(R(A_{1},A_{2},\ldots,A_{n}),dom)$ è uno schema di relazione. $U_{r}$ denota l'insieme dei nomi di attributi di $R$, ovvero \important{$U_{r}=\left\{A_{1},A_{2},\ldots,A_{n}\right\}$}.
            \subsection{Definizione: Schema di base di dati}
                Siano $S_{1},S_{2},\ldots,S_{n}$ schemi di relazioni distinti, $\mathcal{S}=\left\{S_{1},S_{2},\ldots,S_{n}\right\}$ è detto schema di base di dati.
            \subsection{Definizione: Tuple e relazione}
                Una tupla $t$ definita su una relazione $R$ è un insieme di funzioni totali che associano all'attributo di nome $A_{i}$ un valore del dominio di tale attributo. Una relazione definita su uno schema di relazione è un insieme finito di tuple definite su tale schema. Tale relazione è anche detta istanza dello schema. $t=\left[A_{1}:v_{1},A_{2}:v_{2},\ldots,A_{n}:v_{n}\right]$ dove $v_{i}\in dom(A_{i}) \text{ con } i=1,\ldots,n$. Notazione: $t[A_{i}]$ indica il valore dell'attributo $A_{i}$(quindi $v_{i}$) nella tupla $t$.
        \section{Valori nulli}
            Un aspetto importante nella modellazione dei dati riguarda il fatto che non sempre sono disponibili tutte le informazioni sulle entità del dominio applicativo che vengono rappresentate nella base di dati. L'approccio adottato è quello di introdurre un valore speciale, detto \important{valore nullo}, il quale denota la mancanza di un valore.\\
            Nella trattazione assumiamo di denotare il valore nullo con il simbolo '\texttt{?}'. Il valore nullo è un valore accettato in tutti i domini.
        \section{Chiavi}
            Una \important{chiave} di una relazione è un insieme di attributi che distingue fra loro le tuple della relazione. Più formalmente:
            \subsection{Definizione: Chiave e super-chiave}
                Sia $R$ uno schema di relazione. Un insieme $X\subseteq U_{R}$ di attributi di $R$ è chiave di $R$ se verifica le seguenti proprietà:
                \begin{enumerate}
                    \item \important{Univocità}: nella relazione non ci possono essere due tuple distinte che abbiano lo stesso valore per tutti gli attributi della chiave $X$.
                    \item Nessun \textbf{sottoinsieme proprio} di $X$ verifica la proprietà (1).
                \end{enumerate}
                Un insieme di attributi che verifica la proprietà (1) ma non la (2) è detto \important{super-chiave} di $R$. Una super-chiave può essere una chiave della relazione.\\
                In una relazione ci possono essere più insiemi di attributi che soddisfano le due proprietà. In tal caso si parla di \important{chiavi candidate}. \textbf{Una relazione ha sicuramente almeno una chiave (sia primaria che super)}. Nel caso in cui ci sono più chiavi candidate, una di queste viene scelta come \important{chiave primaria} su cui il DBMS ottimizza le operazioni.\\
                Un criterio per scegliere la chiave primaria è quello di scegliere la chiave più piccola in termini di numero di attributi o quella più usata nelle interrogazioni. Una chiave non può contenere valori nulli.
            \subsection{Definizione: Chiave esterna}
                Sia $R_{1}$ ed $R_{2}$ due relazioni, sia $X$ una chiave per $R_{1}$ e $Y$ una chiave per $R_{2}$ tale che $Y$ e $X$ contengano lo stesso numero di attributi e di dominio compatibile (\textit{es. interi e reali sono compatibili}). $X$ è una chiave esterna di $R_{1}$ su $R_{2}$ se per ogni tupla $t$ di $R_{1}$ esiste una tupla $t'$ di $R_{2}$ tale che $t[X]=t'[Y]$. $R_{1}$ viene detta relazione \textbf{referente} e $R_{2}$ relazione \textbf{riferita}.\\
                \remark{Vincolo di integrità refernziale}: se una tupla $t$ di $R_{1}$ fa riferimento ai valori della chiave di una tupla $t'$ di $R_{2}$, allora $t'$ deve esistere in $R_{2}$. Può essere violata da inserimenti e modifiche nella relazione referente e da cancellazioni e modifiche nella relazione riferita.\\
                Una relazione può contenere più chiavi esterne, eventualmente anche sulla stessa relazione e possono assumere valori nulli.\\
                \textbf{Notazione:}
                \begin{equation*}
                    R_{1}\left(\dots,chiave\_esterna^{R_{2}},\dots\right)
                \end{equation*}
\end{document}